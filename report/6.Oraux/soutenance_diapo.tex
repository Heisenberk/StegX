  \documentclass{beamer}
  \usepackage[utf8]{inputenc}
  \usetheme{Warsaw}
  \graphicspath{images/}

  \title{Stéganographie \& Stéganalyse : Soutenance}
  \author{StegX}
  \institute{UFR des Sciences Versailles - L3 Informatique}
  \date{1er Juin 2018}

  \begin{document}

  \begin{frame}
  \titlepage
  \end{frame}
  
  %YASSIN
  \section{Introduction et définitions}
  
  \begin{frame}
  \hspace{5cm}
  \includegraphics[scale=0.7]{pictures/definition1.png}
  \end{frame}
  
  \begin{frame}
  \hspace{5cm}
  \includegraphics[scale=0.7]{pictures/definition2.png}
  \end{frame}
  
  
  \section{Fonctionnement} 
  
  \subsection{Description générale des modules}
  
  \begin{frame}
  \hspace{1.5cm}
  \includegraphics[scale=0.35]{pictures/organigramme_vide.png}
  \end{frame}
  
  \begin{frame}
  \hspace{1.5cm}
  \includegraphics[scale=0.35]{pictures/organigramme_insertion.png}
  \end{frame}
  
  \begin{frame}
  \hspace{1.5cm}
  \includegraphics[scale=0.25]{pictures/organigramme_extraction.png}
  \end{frame}
  
  \begin{frame} %1
	\begin{block}{Description des modules}
	\begin{itemize}
	\setbeamertemplate{itemize item}[circle]
	\item \textcolor{gray} {Vérification de la compatibilité des fichiers.}
	\item \textcolor{gray} {Proposition des algorithmes de stéganographie.}
	\item \textcolor{gray} {Insertion des données.}
	\item \textcolor{gray} {Détection de l'algorithme de stéganographie.}
	\item \textcolor{gray} {Extraction des données.}
	\end{itemize}
	\end{block}
	
	\begin{exampleblock}{Exemple} 
	Alice choisit de cacher \textcolor{red}{message.txt} dans un fichier 
	BMP \textcolor{red}{photo.bmp} et veut obtenir le fichier 
	\textcolor{red}{piece\_jointe.bmp} après la 
	dissimulation. Enfin, elle choisit le mot de passe \textcolor{red}{alicebob}. 
	\end{exampleblock}
  \end{frame}
  
  \begin{frame} %1
	\begin{block}{Description des modules}
	\begin{itemize}
	\setbeamertemplate{itemize item}[circle]
	\item \textcolor{red}{Vérification de la compatibilité des fichiers.}
	\item \textcolor{gray} {Proposition des algorithmes de stéganographie.}
	\item \textcolor{gray} {Insertion des données.}
	\item \textcolor{gray} {Détection de l'algorithme de stéganographie.}
	\item \textcolor{gray} {Extraction des données.}
	\end{itemize}
	\end{block}
	
	\begin{exampleblock}{Exemple} 
	Avec la vérification de la compatibilité des fichiers, il s'agit d'un 
	format \textcolor{red}{BMP non compressé}.  
	\end{exampleblock}
  \end{frame}
  
  \begin{frame} %2
	\begin{block}{Description des modules}
	\begin{itemize}
	\setbeamertemplate{itemize item}[circle]
	\item Vérification de la compatibilité des fichiers.
	\item \textcolor{red}{Proposition des algorithmes de stéganographie.}
	\item \textcolor{gray} {Insertion des données.}
	\item \textcolor{gray} {Détection de l'algorithme de stéganographie.}
	\item \textcolor{gray} {Extraction des données.}
	\end{itemize}
	\end{block}
	
	\begin{exampleblock}{Exemple} 
	\begin{itemize}
	\setbeamertemplate{itemize item}[circle]
	\item Grâce au module Proposition des algorithmes de stéganographie, les 
	spécificités du format de piece\_jointe.bmp ont été déduites. 
	\item Les algorithmes \textcolor{red}{EOF}, \textcolor{red}{LSB} et 
	\textcolor{red}{Metadata} sont proposés. 
	\item Alice choisit l'algorithme \textcolor{red}{EOF}. 
	\end{itemize}
	
	\end{exampleblock}
  \end{frame}
  
  \begin{frame} %3
	\begin{block}{Description des modules}
	\begin{itemize}
	\setbeamertemplate{itemize item}[circle]
	\item Vérification de la compatibilité des fichiers.
	\item Proposition des algorithmes de stéganographie.
	\item \textcolor{red} {Insertion des données.}
	\item \textcolor{gray} {Détection de l'algorithme de stéganographie.}
	\item \textcolor{gray} {Extraction des données.}
	\end{itemize}
	\end{block}
	
	\begin{exampleblock}{Exemple} 
	Après que Alice ait fini la dissimulation, Bob va déduire les spécificités
	du fichier piece\_jointe.bmp et déduire les informations sur le fichier 
	caché message.txt. 
	\end{exampleblock}
  \end{frame}
  
  \begin{frame} %4
	\begin{block}{Description des modules}
	\begin{itemize}
	\setbeamertemplate{itemize item}[circle]
	\item Vérification de la compatibilité des fichiers.
	\item Proposition des algorithmes de stéganographie.
	\item Insertion des données.
	\item \textcolor{red} {Détection de l'algorithme de stéganographie.}
	\item \textcolor{gray} {Extraction des données.}
	\end{itemize}
	\end{block}
	
	\begin{exampleblock}{Exemple} 
	Lors de l'insertion de Alice, l'algorithme EOF sera utilisé où 
	les données de l'hôte, la signature StegX suivies des données du fichier 
	à cacher seront écrites dans piece\_jointe.bmp. 
	\end{exampleblock}
  \end{frame}
  
  \begin{frame} %5
	\begin{block}{Description des modules}
	\begin{itemize}
	\setbeamertemplate{itemize item}[circle]
	\item Vérification de la compatibilité des fichiers.
	\item Proposition des algorithmes de stéganographie.
	\item Insertion des données.
	\item Détection de l'algorithme de stéganographie.
	\item \textcolor{red}{Extraction des données.}
	\end{itemize}
	\end{block}
	
	\begin{exampleblock}{Exemple} 
	L'algorithme EOF sera utilisé pour extraire les  données du fichier 
	caché de message.txt. 
	\end{exampleblock}
  \end{frame}
  	

  
  \subsection{Signature StegX}
  
  \begin{frame}
  
  \begin{block}{Signature StegX}
	\begin{itemize}
	\setbeamertemplate{itemize item}[circle]
	\item Identificateur de la méthode (1 octet).
	\item Identificateur de l'algorithme (1 octet). 
	\item Taille du fichier caché (4 octets). 
	\item Taille du nom du fichier caché (1 octet). 
	\item Nom du fichier caché (entre 1 et 255 octets). 
	\item Mot de passe (64 octets).  
	\end{itemize}
	\end{block}
  
  \includegraphics[scale=0.55]{pictures/signature_6.png}
  \end{frame}

  
  
  \section{Description des algorithmes de stéganographie} %OK
  
	% LSB OK
	\subsection{Least Significant Bit (LSB)}
	\begin{frame}
  
	\begin{block}{Least Significant Bit}
	\begin{itemize}
	\setbeamertemplate{itemize item}[circle]
	\item Modification des bits de poids faible des octets de données de 
	l'hôte. 
	\end{itemize}
	\end{block}
	
	\begin{exampleblock}{Avantage} 
	L'utilisation de cet algorithme n'augmente pas la taille du fichier 
	résultat en fonction de la taille du fichier caché. 
	\end{exampleblock}
	
	\begin{alertblock}{Inconvénient} 
	Le fichier à cacher doit être assez petit pour pouvoir le cacher intégralement 
	dans le fichier hôte. 
	\end{alertblock}
	
	\hspace{3.5cm}
    \includegraphics[scale=0.1]{pictures/lsb.png}
	\end{frame}
  
    % EOF OK
    \subsection{End Of File (EOF)}
    \begin{frame}
    
	\begin{block}{End Of File}
	\begin{itemize}
	\setbeamertemplate{itemize item}[circle]
	\item Écriture des données à cacher après la fin officielle du fichier 
	hôte. 
	\end{itemize}
	\end{block}
	
	\begin{exampleblock}{Avantage} 
	Il n'y a pas de limite de taille pour cacher le fichier. 
	\end{exampleblock}
	
	\begin{alertblock}{Inconvénient} 
	L'utilisation de cet algorithme augmente considérablement la taille du 
	fichier résultat par rapport au fichier hôte. 
	\end{alertblock}
	
	\hspace{1cm}
    \includegraphics[scale=0.2]{pictures/eof.png}
    
    \end{frame}
    
    % METADATA OK
	\subsection{Metadata}
    \begin{frame}
    
	\begin{block}{Metadata}
	\begin{itemize}
	\setbeamertemplate{itemize item}[circle]
	\item Écriture des données à cacher dans des blocs de données spécifiques 
	qui ne modifieront pas les données originales. 
	\end{itemize}
	\end{block}
	
	\begin{exampleblock}{Avantage} 
	Il n'y a pas de limite de taille pour cacher le fichier. 
	\end{exampleblock}
	
	\begin{alertblock}{Inconvénient} 
	L'utilisation de cet algorithme augmente considérablement la taille du 
	fichier résultat par rapport au fichier hôte. 
	\end{alertblock}
	
	\hspace{3.5cm}
    \includegraphics[scale=0.2]{pictures/meta.png}
    
    \end{frame}
    
    %EOC OK
    \subsection{End Of Chunk (EOC)}
    \begin{frame}
    
	\begin{block}{End Of Chunk}
	\begin{itemize}
	\setbeamertemplate{itemize item}[circle]
	\item Écriture des données à cacher après les différents chunks interprétables 
	du fichier hôte. Ces données seront non reconnus et donc ignorés.  
	\end{itemize}
	\end{block}
	
	\begin{exampleblock}{Avantage} 
	Il n'y a pas de limite de taille pour cacher le fichier. 
	\end{exampleblock}
	
	\begin{alertblock}{Inconvénient} 
	Il est possible de créer des distorsions sur le fichier vidéo où 
	l'interprétation pour certains logiciels multimédia peut être fait sur 
	ces données cachées. 
	\end{alertblock}
	
	\hspace{4.3cm}
    \includegraphics[scale=0.08]{pictures/flv_512.png}
    
    \end{frame}
  
	%Junk Chunk OK
    \subsection{Junk Chunk}
    \begin{frame}
    
	\begin{block}{Junk Chunk}
	\begin{itemize}
	\setbeamertemplate{itemize item}[circle]
	\item Écriture des données à cacher dans un chunk appelé "junk" : les 
	données ne seront pas interprétées  
	\end{itemize}
	\end{block}
	
	\begin{exampleblock}{Avantage} 
	Il n'y a pas de limite de taille pour cacher le fichier. 
	\end{exampleblock}
	
	\begin{alertblock}{Inconvénient} 
	L'utilisation de cet algorithme augmente considérablement la taille du 
	fichier résultat par rapport au fichier hôte. 
	\end{alertblock}
	
	\hspace{4.4cm}
    \includegraphics[scale=0.08]{pictures/avi-512.png}
    
    \end{frame}
     
  
  
  \section{Problèmes et points délicats rencontrés} 
  
  \begin{frame}
  
	\begin{block}{Lecture et écriture de fichiers : endianness en fonction des formats}
	\begin{itemize}
	\setbeamertemplate{itemize item}[circle]

	\item Little endian et big endian selon les formats. 
	\item Travail de recherche poussé pour chaque format et pour chaque 
	algorithme de stéganographie.
	\end{itemize}
	\end{block}
		
	\begin{block}{Format MP3}
	\begin{itemize}
	\setbeamertemplate{itemize item}[circle]
	\item Format compressé utilisant des algorithmes de compression complexes.
	\item Étude des versions des formats (MPEG 1 Layer III, MPEG 2 Layer III). 
	\item Étude des versions de formats de métadonnée (ID3 version 1, ID3 version 2). 
	\end{itemize}
	\end{block}
	
	\end{frame}
  
  
  
  \section{Bilan logiciel et humain}
  
  \begin{frame}
  
	\begin{block}{Bilan logiciel}
	\begin{itemize}
	\setbeamertemplate{itemize item}[circle]
	\item Faire de la stéganographie sur des fichiers image, audio et vidéo
	(insertion et extraction). \checkmark
	\item Plusieurs formats sont gérés et une diversité dans les algorithmes 
	sont proposés. \checkmark
	\item Réaliser deux interfaces : ligne de commande et graphique. \checkmark
\setbeamertemplate{itemize item}[triangle]
	\item Futures améliorations : nouveaux formats et nouveaux algorithmes. 

\end{itemize}
	
	\end{block}
	
	\begin{block}{Bilan humain}
	\begin{itemize}
	\setbeamertemplate{itemize item}[circle]
	\item L'équipe s'est efforcée à travailler de façon professionnelle et ordonnée.
	\item Projet de grande envergure qui a nécessité de diviser la conception 
	selon les trois types de formats étudiés.  
	\end{itemize}
	\end{block}

	\end{frame}
  
  \end{document}
  
