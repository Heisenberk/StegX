\documentclass[11pt]{article}

\usepackage[utf8]{inputenc}
\usepackage[T1]{fontenc}
\usepackage[french]{babel}
\usepackage[top=3.2cm, bottom=3.2cm, left=3.2cm, right=3.2cm]{geometry}
\usepackage{hyperref}
\usepackage{graphicx}
\usepackage{epsfig}
\usepackage{array}
\hypersetup{
    colorlinks=true,
    breaklinks=true,
    urlcolor=red,
}
\parskip=5pt

\title{\huge{\textbf Cahier des charges}}
\author{AYOUB Pierre, BASKEVITCH Claire, BESSAC Tristan, \\
CAUMES Clément, DELAUNAY Damien, DOUDOUH Yassin}
\date{Mercredi 14 Mars 2018}

\begin{document}

\maketitle
\vspace{20em}
\begin{center}\includegraphics{pictures/Application.png}\end{center}

\section{Préambule}

\subsection{Définition des termes du sujet}

La stéganographie est l'art de la dissimulation, appliquée en informatique en
cachant des données dans d'autres données. Cette dissimulation se fait
généralement au sein de fichiers multimédias. La stéganographie se différencie
de la cryptographie, qui correspond à chiffrer un message afin qu'il soit
illisible par une personne différente de l'émetteur et du destinataire. En
effet, un message chiffré en cryptographie sera visible par tous mais illisible,
tandis qu'un message caché dans un fichier $f$ en stéganographie ne sera vu que
si un inconnu sait que $f$ contient un message et connaît l'algorithme pour
l'interpréter. 

La stéganalyse, quant à elle, est la recherche de données cachées dans des
fichiers suspects. Si ces données sont identifiées, il faut ensuite réussir à
les extraire pour les lire. Il s'agit donc de la méthode inverse à la
stéganographie. 

\subsection{Historique}

La stéganographie est une méthode très ancienne dont la première référence à
cette utilisation date du premier siècle avant Jésus-Christ. Elle apparaît dans
un récit écrit par Hérodote qui raconte comment deux citoyens communiquaient
secrètement : le premier citoyen rasait la tête de son esclave et lui écrivait
un message sur son crâne. Ensuite, il fallait attendre que les cheveux de
l'esclave repoussent puis envoyer ce dernier chez le deuxième citoyen. Ce
citoyen devait de nouveau raser la tête de l'esclave pour découvrir le message
qui lui était destiné. Une autre utilisation de la stéganographie consistait à
utiliser de l'encre, invisible à l'oeil nu, mais qui était révélée à la chaleur.

Avec l'émergence de l'informatique, les techniques de stéganographie se sont
renouvelées. En effet, il est désormais possible de cacher des données dans
d'autres données. Cette multiplicité de techniques stéganographiques, grâce à
l'informatique, montre l'étendue de cette application dans tous les domaines.
Par exemple, la stéganographie moderne a été utilisée dans des communications
terroristes (transmission de messages) ou dans les signatures de fichiers
multimedia (tatouage numérique) afin de protéger les droits d'auteurs. 

\section{Conducteurs du projet}

\subsection{But du projet}

Le but du projet est de réaliser un logiciel de stéganographie permettant à des
personnes lambdas de communiquer sans que l'on soupçonne que leurs
communications soient en réalité compromettantes. Le but de l'application est
de permettre à un utilisateur $U_1$ d'envoyer des données cachées à un autre
utilisateur $U_2$. Ce deuxième utilisateur devra pouvoir interpréter ces données
en utilisant la même application que $U_1$. 

\subsection{Motivation du projet}

La motivation du projet est venue par notre envie de la majorité des membres de
ce groupe de projet d'obtenir le master SeCReTs. En effet, nous voulions tous
réaliser un projet en rapport à la cryptographie et c'est donc pour cela que
nous nous sommes réunis afin de réaliser ce type de projet. 

\section{Contraintes du projet}

\subsection{Calendrier}

Le calendrier est imposé et suit les étapes suivantes : 

\begin{itemize}
\item Le cahier des charges doit être remis le 14 mars.
\item Le cahier des spécifications est à remettre le 18 avril.
\item La remise du produit au client est le 25 mai.
\item La présentation du produit au client sera le 1 juin.
\end{itemize}


\subsection{Contraintes imposées}

Plusieurs contraintes sont imposées par le client : 

\begin{itemize}
\item Le produit permettra à celui qui l'utilise de cacher des données dans des
    fichiers de différents types : les images, le son et la vidéo seront pris
    en charge. Les formats seront choisis par le concepteurs de
    l'application. 
\item Le fichier à analyser est le fchier hôte : on sait à l'avance que celui-ci
    contient des données à extraire et il faut donc les prélever. 
\item L'application aura une interface graphique : elle sera manipulée par les
    utilisateurs voulant découvrir un message envoyé par quelqu'un ayant
    utilisé cette même application, ou voulant cacher des données dans un
    fichier. Il devra néanmoins choisir un fichier hôte que l'application
    puisse manipuler. 
\item Le logiciel proposera également une interface en ligne de commande, qui
    proposera les mêmes fonctionnalités que l'interface graphique. 
\end{itemize}

\section{Exigences fonctionnelles}

\subsection{Portée du produit}

L'application sera utilisée par des utilisateurs qui pourront tous faire les
mêmes actions. La première consiste en la dissimulation des données dans un
fichier, tandis que la seconde consiste en l'extraction des données cachées
depuis le fichier hôte. Cela permettra à un utilisateur d'insérer des données à
cacher dans un fichier, afin de l'envoyer à un autre utilisateur. Ce dernier va
ensuite faire la tâche inverse pour les récupérer.

\subsection{Exigences du client}

L'application doit respectée deux exigences pour le client : elle doit permettre
à un utilisateur ne connaissant pas la stéganographie de pouvoir facilement
utiliser toutes les fonctionnalités de l'application grâce à une interface
graphique intuitive. De plus, une interface en ligne de commande doit être
également proposée pour les utilisateurs sachant manipuler un programme dans un
terminal. Cela permetra à ces utilisateurs plus exprimentés d'intégrer
l'application dans des scripts et de l'utiliser conjointement à d'autre
programme.

\section{Exigences non fonctionnelles}

\subsection{Apparence et perception}

Le logiciel doit permettre à n'importe quel utilisateur de pouvoir cacher ses
données dans des fichiers. En effet, la facilité d'utilisation de l'application
sera ciblée pour le développement. Il sera facile de naviguer entre les deux
menus en fonction de si l'utilisateur veut cacher ses données ou s'il veut
extraire les données cachées d'un fichier. Il pourra naviguer dans son système
de gestion de fichiers afin de choisir le fichier hôte et quel fichier
contiendra ces données cachées. 

\subsection{Performance}

L'application devra être rapide pour l'utilisateur qui s'en sert. Bien entendu,
la stéganographie sur certains fichiers lourds (tels que des fichiers vidéos)
rendra l'exécution plus lente mais elle ne devra pas être trop importante pour
l'utilisateur. 

\subsection{Exigences culturelles, politiques et légales}

L'application de stéganographie vise des clients en France. Il faut donc
respecter les lois françaises concernant l'utilisation de cette application. La
loi du 21 juin 2004, pour la confiance dans l'économie numérique, définit cette
application comme moyen de cryptologie car elle vise à transformer des données
pour garantir la sécurité de la transmission de celles-ci. Par ailleurs,
l'article 30 de cette même loi oblige la déclaration de l'application si cette
dernière est importée et/ou exportée. 

\section{Modules du produit}

\subsection{Organigramme}

\newpage

\includegraphics[scale=0.55]{pictures/Organigramme.png}

\paragraph{Liste des modules et de leurs fonctionnalités :}

\begin{description}

\item[a)] \textbf{Interface graphique / Interface en ligne de commande} :
    interfaces permettant à l'utilisateur de choisir parmi les deux
    fonctionnalités possibles de l'application. Il peut dissimuler des
    données dans un fichier (dont le type et le format sont pris en charge
    par l'application). Ou bien, il peut extraire les données cachées d'un
    fichier. 

\item[b)] \textbf{Compatibilité} : le format du fichier hôte (pour le module
    \textit{Dissimulation de données}) ou le format du fichier à analyser (pour
    le module \textit{Extraction de données cachées}), choisis par
    l'utilisateur, est vérifié pour savoir s'il est bien pris en charge
    par l'application. 

\item[c)] \textbf{Proposition des algorithmes de stéganographie} : en fonction
    du type et du format du fichier hôte, ainsi que de la taille des données à
    cacher, un ou plusieurs algorithmes seront proposés. 

\item[d)] \textbf{Détection de l'algorithme de stéganographie} : analyse du
    fichier pour découvrir quel algorithme a été utilisé afin de les extraire
    correctement par la suite. 

\item[e)] \textbf{Insertion des données} : la copie des données du fichier hôte
    sera modifiée avec l'insertion des données à cacher à l'aide de l'algorithme
    choisi par l'utilisateur. 

\item[f)] \textbf{Extraction} : les données cachées dans le fichier à analyser
    sont extraites. 

\end{description}

\paragraph{Liste des informations qui circulent entre les modules :}

\begin{description}

\item[1)] 
\begin{itemize}
\item Utilisation de l'application (dissimulation ou extraction).
\item Nom du fichier hôte, nom du fichier à cacher et chemin du fichier à créer
    (pour la dissimulation).
\item Nom du fichier à analyser et chemin du fichier résultant de l'extraction
    des données cachées (pour l'extraction).
\end{itemize}
\item[2)] 
\begin{itemize}
\item Nom du fichier hôte.
\item Nom du fichier à cacher.
\item Chemin du fichier à créer qui dissimulera les données à cacher et aura
    l'apparence du fichier hôte.
\end{itemize}
\item[3)] 
\begin{itemize}
\item Nom du fichier contenant les données cachées à analyser.
\item Chemin du fichier résultant de l'extraction des données cachées.
\end{itemize}
\item[4)] 
\begin{itemize}
\item Nom du fichier hôte et nom du fichier à cacher (pour la dissimulation).
\item Nom du fichier contenant les données cachées à analyser (pour
    l'extraction).
\end{itemize}
\item[5)]
\begin{itemize}
\item Fichier hôte et fichier à cacher (pour la dissimulation).
\item Fichier contenant les données cachées à analyser (pour l'extraction).
\end{itemize}
\item[6)]
\begin{itemize}
\item Fichier hôte.
\item Fichier à cacher.
\item Chemin du fichier à créer qui dissimulera les données à cacher.
\end{itemize}
\item[7)]
\begin{itemize}
\item Fichier contenant les données cachées à analyser.
\item Chemin du fichier résultant de l'extraction des données cachées.
\end{itemize}
\item[8)]
\begin{itemize}
\item Liste des algorithmes que l'utilisateur peut utiliser (selon le format du
    fichier hôte et la taille des données à cacher).
\end{itemize}
\item[9)]
\begin{itemize}
\item Choix de l'algorithme par l'utilisateur.
\item Mot de passe choisi par l'utilisateur (s'il a choisis l'option de protéger
    ses données par mot de passe).
\end{itemize}
\item[10)]
\begin{itemize}
\item Choix de l'algorithme détecté dans le fichier à analyser.
\end{itemize}
\item[11)]
\begin{itemize}
\item Mot de passe pour extraire les données (s'il a choisis l'option de
    protéger ses données par mot de passe).
\end{itemize}
\item[12)]
\begin{itemize}
\item Fichier hôte.
\item Fichier à cacher.
\item Chemin du fichier à créer lors de la dissimulation de données.
\item Nom de l'algorithme de stéganographie utilisé.
\end{itemize}
\item[13)]
\begin{itemize}
\item Fichier contenant les données cachées à analyser.
\item Chemin du fichier résultant de l'extraction des données cachées.
\item Nom de l'algorithme détecté.
\end{itemize}
\item[14)]
\begin{itemize}
\item Données de l'hôte où les données à cacher ont été insérées en utilisant
    l'algorithme de stéganographie.
\end{itemize}
\item[15)]
\begin{itemize}
\item Données cachées extraites du fichier hôte.
\end{itemize}
\end{description}

\subsection{Algorithmes proposés}

L'application permettra de cacher des données dans des fichiers de différents
types (image, son, video). Plusieurs méthodes seront utilisées : 

\subsubsection{Algorithme LSB (Least Significant Bit)}

L'algorithme LSB permet de cacher des bits dans des octets tel qu'ils seront
invisibles pour l'Homme. Il permet de cacher des données dans un fichier sans en
altérer sa taille. Dans le cas de la stéganographie sur une image non compressé,
chaque pixel d'une image correspond à un triplet de nombres dit triplet RGB. Ce
triplet correspond aux composantes de couleurs Rouge-Vert-Bleu, prenant des
valeurs allant de 0 à 255 si un pixel est codé sur 24 bits. Le but de cet
algorithme est donc de cacher des bits dans cette image. Pour se faire, nous
allons remplacer les 2 bits de poids faibles de chaque composante des pixels de
l'image. En effet, à l'oeil nu, l'homme ne discernera jamais le changement
minime de couleur induis par la modification.  Prenons un exemple de couleur
$C_1$ dont le triplet est $(219,27,91)$. 

R : $219_{10} = 11011011_2$ \qquad G : $27_{10} = 00011011_2$
\qquad B : $91_{10} = 01011011_2$

Imaginons que la donnée à cacher dans le fichier composé de cet unique pixel de
couleur $C_1$ correspond à la suite de bits $B = 000000_2$. Après modification
des 2 bits de poids faibles de chaque composants du pixel, cela donne une
autre couleur $C_2$ défini par le triplet suivant : 

R : $216_{10} = 11011000_2$ \qquad G : $24_{10} = 00011000_2$
\qquad B : $88_{10} = 01011000_2$

La figure suivantes illustres les deux couleurs $C_1$ et $C_2$ côte à côte,
montrant ainsi qu'un humain ne pourra jamais détecter un changement de bit : 

\begin{figure}[h]
 \begin{minipage}{.46\linewidth}
  \centering\epsfig{figure=pictures/219_27_91.png}
  \caption{Couleur $C_1$}
 \end{minipage} \hfill
 \begin{minipage}{.46\linewidth}
  \centering\epsfig{figure=pictures/216_24_88.png}
  \caption{Couleur $C_2$}
 \end{minipage}
\end{figure}

Cet algorithme peut également s'appliquer à d'autres types de fichiers. Pour le
son non compressé, il est possible de très peu modifier l'amplitude de chaque
échantillon de son, de telle manière que l'on peut y cacher des données sans
altérer le son original pour la perception humaine. Pour la vidéo non compressé,
un tel fichier est composé d'un enchaînement de multiples frames (images) et il
est donc possible de manipuler les données pour pouvoir en cacher d'autres, de
la même manière que pour les deux autres types de fichier cités précédemment.

Pour la partie réception du fichier, il faut connaître combien de bits sont
cachés  Il faudra donc calculer la taille maximale du message à cacher qui sera
une puissance de 2. En effet, en fonction de la taille du fichier, un certain
nombre de bits sera réservé pour connaître la taille des données à cacher. En
fonction de ces informations, la suite de bits cachée sera donc formée.

\subsubsection{Algorithme EOF (End Of File)}

Chaque format de fichier a une mise en forme unique permettant de décrire le
type de données souhaité. L'entête du fichier va contenir la signature du
format, appelé le Magic Number, ainsi que plusieurs autres octets décrivant ce
fichier. Le reste du fichier contient le contenu visible par l'utilisateur grâce
à l'application appropriée. 

Pour que l'application utilisée par l'utilisateur sache quand la lecture du
fichier doit s'arrêter, certains formats de fichier continnent, à la toute fin,
un octet représentant la fin du fichier (EOF). Si des données existent après ce
EOF, elles ne seront pas interprétées par l'application. L'algorithme EOF est un
algorithme très utilisé dans la stéganographie pour cacher des données : il
consiste à écrire une suite de bits représentant les données à cacher
directement à la fin du fichier hôte.

En revanche, cet algorithme souffre de nombreuses faiblessses : les données
cachées sont facilement identifiable dans une stéganalyse et la taille du
fichier hôte se voit incrémenté de la taille des données à cacher.

\subsubsection{Méthode manipulant les métadonnées}

Les fichiers multimédias, tels que les images, les sons et les vidéos,
contiennent plusieurs types de données. Tout d'abord, la majeure partie du
fichier représente les données du fichier en elles-mêmes. Dans le cas d'une
image, on pourrait donner l'exemple des données représentant chaque pixel de
l'image. 

Mais souvent, en amont des données principales du fichier, il existe les
métadonnées servant à décrire le fichier représenté. Par exemple, pour une
image, il est utile de stocker la taille en pixels de l'image afin que la
visionneuse d'image sache la représenter correctement. Il y a aussi des zones du
fichier réservées à l'utilisateur afin qu'il y mette des << commentaires >>, par
exemple pour représenter l'origine du fichier ou encore des informations sur la
prise de vue dans le cas d'une photo. Les métadonnées sont affichés si
l'utilisateur le demande, mais sont en général cachés par défaut.

De ce fait, les métadonnées sont très utilisées en stéganographie. Elles
permettent d'insérer des données qui seront à premère vue invisible pour
l'utilisateur. Pour pouvoir cacher des données en manipulant les métadonnées,
il faut donc analyser chaque format pris en charge par l'application afin de
connaître les détails des différents morceaux de données présents dans le
fichier. 

\subsection{Estimations des coûts}
\begin{tabular}{|c|c|c|c|}
  \hline
  \textbf{Module} & \textbf{Coût en nombre} & \textbf{Coût en} & \textbf{Personnel(s) en charge} \\
   \textbf{de l'application} & \textbf{de lignes} & \textbf{temps} & \textbf{du module} \\
  \hline
  Stéganographie des & X & X & CAUMES Clément \& \\
   fichiers images & lignes & heures & DOUDOUH Yassin \\
  \hline
  Stéganographie des & X & X & AYOUB Pierre \& \\
   fichiers audios & lignes & heures & DELAUNAY Damien \\
  \hline
  Stéganographie des & X & X & BASKEVITCH Claire \& \\
   fichiers vidéos & lignes & heures & BESSAC Tristan \\
  \hline
    Interface en ligne & X & X & X \\ 
    de commande & lignes & heures & \\
  \hline
  Interface graphique & X lignes & X heures & X \\
  \hline
\end{tabular}

\section{Autres aspects du projet}

\subsection{Solutions sur étagère déjà existantes}

Plusieurs applications de stéganographie existent déjà, permettant de cacher des
données dans différents formats de fichiers multimédia. Pourtant, parmis les 23
applications de stéganographie que nous avons recensés, seulement une seule
propose de cacher des données à la fois dans des images, du son et de la vidéo.

\subsection{Tâches à réaliser pour le développement de l'application}

\begin {enumerate}
\item Identification du produit : étude des volontés et des demandes du client
    (énoncé du projet 24/01/18).
\item Etude du produit : recherche des outils et des algorithmes pour répondre
    au produit demandé par le client ; devis livré au client avec l'estimation
    des coûts du produit (cahier des charges 14/03/18).
\item Mise en relation avec le client : présentation du produit, de ses
    différents modules et fonctionnalités (présentation orale 21/03/18).
\item Etude spécifique du projet : identification précise des méthodes utilisées
    pour répondre aux demandes du client (spécifications 18/04/18).
\item Remise du produit au client : finalisation du produit, rendu du manuel
    d'utilisation et du compte-rendu (remise du compte rendu 25/05/18).
\item Démonstration du produit devant le client : explications fonctionnelles du
    produit (soutenance 01/06/18).
\end{enumerate}

\subsection{Améliorations pour les versions futures du projet}

La stéganographie se distingue de la cryptographie par le fait que, dans l'un,
le message caché est visible par tous si celui-ci est extrait ; tandis que, dans
l'autre, le message est transmis en étant chiffré sur un canal non-sûr. 

Pour une projection à long terme, dans d'éventuelles versions du logiciel, nous
pourrions améliorer l'application en chiffrant les données cachées. En effet,
lors de l'interception d'un éventuel fichier cachant des données, il faudra, en
plus de les extraire, les déchiffrer : ce qui rend la tâche beaucoup plus longue
pour celui qui intercepte le fichier et qui tente de récupérer ces données
cachées. 

De plus, pour la versatilité de l'application, nous pourrions prendre en charge
de nouveaux formats de fichiers. Cela permettrait d'augmenter la portée du
logiciel. La manipulation de données compressées serait également une
amélioration conséquente. 

\subsection{Choix du langage et de l'interface}

TODO.

\section{Conclusion}

Après l'étude des demandes du client, nous mettrons en avant l'application
\textbf{StegX}. Cette application sera implémentée par un groupe de six
étudiants en Licence Informatique de l'Université de Versailles
Saint-Quentin-en-Yvelines (UVSQ), ayant l'ambition d'obtenir un master en
Cryptographie et Sécurité informatique (SeCReTs). 

\section{Bibliographie}

\paragraph{Méthode Volere}
\begin{itemize}
\item QualityStreet.fr - Spécifications, Exigences et Cahier des charges : \\
    \url{http://www.qualitystreet.fr/2007/07/02/specifications-exigences-et-cahier-des-charges-comment/}
\item Volere.co.uk - Plan de cahier des charges et spécification des exigences
    non fonctionnelles avec Volere : \\
    \url{http://volere.co.uk/pdf%20files/template_fr.pdf}
\end{itemize}

\paragraph{Diagramme de Gantt}
\begin{itemize}
\item Gantt.com - Qu'est-ce qu'un diagramme de Gantt : \\
    \url{http://www.gantt.com/fr/}
\item Commentçamarche.com - GANTT - Diagramme de GANTT : \\
    \url{http://www.commentcamarche.com/contents/982-gantt-diagramme-de-gantt}
\end{itemize}

\paragraph{Stéganographie générale}
\begin{itemize}
\item Eric Cole (Wiley) : Hiding in Plain Sight, Steganography and the Art of Covert Communication
\item Michael Raggo, Chet Hosmer (Elsevier) : Data Hiding, Exposing Concealed
    Data in Multimedia, Operating Systems, Mobile Devices and Network Protocols
\item Andrew D. Ker (Oxford University) - Information Hiding : \\
    \url{https://www.cs.ox.ac.uk/andrew.ker/docs/informationhiding-lecture-notes-ht2016.pdf}
\item SANS Institute - Steganography, The Right Way : \\
    \url{https://www.sans.org/reading-room/whitepapers/stenganography/steganography-1584}
\item SANS Institute - A Detailed look at Steganographic Techniques and their use in an Open-Systems Environment : \\
    \url{https://www.sans.org/reading-room/whitepapers/covert/detailed-steganographic-techniques-open-systems-environment-677}
\item BibMath.net - La stéganographie : \\
    \url{http://www.bibmath.net/crypto/index.php?action=affiche&quoi=stegano/index}
\item SecuritéInfo.com - La stéganographie - L'art de la dissimulation de
    données : \\
    \url{https://www.securiteinfo.com/attaques/divers/steganographie.shtml}
\end{itemize}

\paragraph{Stéganographie sur les fichiers images}
\begin{itemize}
    \item Dalia Battikh (INSA de Rennes) - Sécurité de l’information par
    stéganographie basée sur les séquences chaotiques : \\
    \url{https://tel.archives-ouvertes.fr/tel-01275346/document}
\end{itemize}

\paragraph{Stéganographie sur les fichiers images JPEG}
\begin{itemize}
\item Daniel L. Currie, Cynthia E. Irvine - Surmounting the Effects of Lossy
    Compression on Steganography : \\
    \url{https://csrc.nist.gov/csrc/media/publications/conference-paper/1996/10/22/proceedings-of-the-19th-nissc-1996/documents/paper014/stegox.pdf}
\item Matus Jokay, Tomas Moravcik (Tatra Mountains Mathematical Publications) -
    Image-Based JPEG Steganography : \\
    \url{https://www.sav.sk/journals/uploads/0317153109jo-mo.pdf}
\item ScienceDirect.com - Data hiding inside JPEG images with high resistance to
    steganalysis using a novel technique, DCT-M3 : \\
    \url{https://www.sciencedirect.com/science/article/pii/S209044791730031X}
\end{itemize}

\paragraph{Stéganographie sur les fichiers audios}
\begin{itemize}
\item International Journal of Multimedia and its Applications (IJMA) -
    Information Hiding Using Audio Steganography, A Survey : \\
    \url{http://aircconline.com/ijma/V3N3/3311ijma08.pdf}
\end{itemize}

\paragraph{Stéganographie sur les fichiers audios MP3}
\begin{itemize}
\item AppliedTech.IIT.edu - MP3 Steganography : \\
    \url{https://appliedtech.iit.edu/school-applied-technology/projects/mp3-steganography}
\item School of Applied Tech at Illinois Tech - Mp3 Steganography, Presented at
    Forensecure (Cyber Forensics \& Security Conference 2016) : \\
    \url{https://www.youtube.com/watch?v=57SHhsKvk08}
\item Universiti Teknologi Malaysia - MP3 Steganography Review : \\
    \url{http://www.ijcsi.org/papers/IJCSI-9-6-3-236-244.pdf}
\end{itemize}

\paragraph{Stéganographie sur les fichiers audios WAVE-PCM}
\begin{itemize}
\item CodeProject.com - Steganography VIII - Hiding Data in Wave Audio Files : \\
    \url{https://www.codeproject.com/articles/6960/steganography-viii-hiding-data-in-wave-audio-files}
\end{itemize}

\paragraph{Stéganographie sur les fichiers vidéos}
\begin{itemize}
    \item International Journal of Engineering and Innovative Technology (IJEIT)
    - Improved Protection In Video Steganography Using DCT \& LSB : \\
    \url{http://www.ijeit.com/vol%201/Issue%204/IJEIT1412201204_07.pdf}
\end{itemize}

\paragraph{Stéganographie sur les fichiers vidéos FLV}
\begin{itemize}
    \item Jason Cruz, Nathaniel Libatique, Gregory Tangon (Ateneo de Manila
    University) - Steganography and Data Hiding in Flash Video (FLV) : \\
    \url{https://ateneo.edu/sites/default/files/Steganography%20and%20data%20hiding%20in%20flash%20video%20%28FLV%29_0.pdf}
\end{itemize}

\paragraph{Outils de Stéganographie}
\begin{itemize}
\item PedramHayati.com - A Survey of Steganographic and Steganalytic Tools for the Digital Forensic Investigator : \\
    \url{http://www.pedramhayati.com/images/docs/survey_of_steganography_and_steganalytic_tools.pdf}
\end{itemize}

\paragraph{Stéganalyse sur les fichiers vidéos}
\begin{itemize}
\item PeerJ.com - Forensic analysis of video steganography tools : \\
    \url{https://peerj.com/articles/cs-7/}
\end{itemize}

\end{document}
