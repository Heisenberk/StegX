\documentclass[11pt]{article}
%\documentclass{book}
\usepackage[utf8]{inputenc}
\usepackage[T1]{fontenc}
\usepackage[french]{babel}
\usepackage[top=1.8cm, bottom=1.8cm, left=1.8cm, right=1.8cm]{geometry}
\usepackage[linktocpage,colorlinks=false]{hyperref}
\usepackage{graphicx}
\usepackage{epsfig}
\usepackage{amssymb}
\usepackage{amsmath}
\usepackage{array}
\usepackage{subfig}
\usepackage{multicol}
\usepackage{caption}
\usepackage{listings}
\usepackage{algorithm}
\usepackage{algorithmic}
\hypersetup{
    colorlinks=true,
    breaklinks=true,
    urlcolor=red,
}
\parskip=5pt

\title{\huge{\textbf Compte Rendu}}
\author{AYOUB Pierre, BASKEVITCH Claire, BESSAC Tristan, \\
CAUMES Clément, DELAUNAY Damien, DOUDOUH Yassin}
\date{Mercredi 18 Avril 2018}

\begin{document}

\maketitle
\vspace{20em}
\begin{center}\includegraphics{pictures/Application.png}\end{center}
\newpage

\tableofcontents

\newpage

\section{Introduction}

Le cahier des charges ainsi que les spécifications ont été réalisés par 
l'équipe de StegX, il est maintenant temps d'implémenter le logiciel.
L'application StegX proposera à ses utilisateurs de manipuler une interface 
graphique ou en ligne de commandes afin de cacher des données dans d'autres 
données (pour des formats pris en charge par StegX). De plus, StegX permettra 
d'extraire des données d'un fichier que l'on considère déjà comme cachant 
des données. Le logiciel proposera plusieurs algorithmes de stéganographie
tels que EOF, LSB et Metadata. 

En plus de l'implémentation, il sera utile de présenter le fonctionnement 
de l'architecture de l'application avec les explications techniques. 
Une description des points délicats seront mis en valeur ainsi qu'un 
bilan technique sur l'application et humain sur l'équipe de conception. 

\section{Architecture du produit}

\section{Explications techniques}

\section{Description du fonctionnement}

\section{Description des points délicats de la programmation}

\section{Comparaison estimation - implémentation}

\small
\hspace{-1cm}
\begin{tabular}{|c|c|c|c|}
  \hline
  \textbf{Module de l'application} & \textbf{Coût en nombre de lignes} & \textbf{Coût en temps} & \textbf{Personnel(s) en charge} \\
  \hline
    Interface en ligne & Estimation : 200 lignes & Estimation : 30 heures & BASKEVITCH Claire \& \\ 
    de commande & Implémentation : X lignes & Implémentation : X heures & BESSAC Tristan \\
  \hline
  Interface & Estimation : 400 lignes & Estimation : 30 heures & AYOUB Pierre \& \\
  graphique & Implémentation : X lignes & Implémentation : X heures & DELAUNAY Damien \\
  \hline
  Vérification de la & Estimation : 300 lignes & Estimation : 30 heures& CAUMES Clément \& \\
   compatibilité des fichiers & Implémentation : X lignes & Implémentation : X heures & DOUDOUH Yassin \\
  \hline
    Proposition des algos & Estimation : 100 lignes & Estimation : 15 heures & CAUMES Clément \& \\
   de stéganographie & Implémentation : X lignes & Implémentation : X heures & DOUDOUH Yassin \\
  \hline
    Détection de l'algo & Estimation : 100 lignes & Estimation : 15 heures & AYOUB Pierre \& \\
   de stéganographie & Implémentation : X lignes & Implémentation : X heures & DELAUNAY Damien \\
  \hline
  Dissimulation \& Extraction & Estimation : 250 lignes & Estimation : 40 heures & CAUMES Clément \& \\
   fichiers images & Implémentation : X lignes & Implémentation : X heures & DOUDOUH Yassin \\
  \hline
  Dissimulation \& Extraction & Estimation : 250 lignes & Estimation : 40 heures & AYOUB Pierre \& \\
   fichiers audios & Implémentation : X lignes & Implémentation : X heures & DELAUNAY Damien \\
     \hline
  Dissimulation \& Extraction & Estimation : 250 lignes & Estimation : 40 heures & BASKEVITCH Claire \& \\
   fichiers vidéos & Implémentation : X lignes & Implémentation : X heures & BESSAC Tristan \\
  \hline
\end{tabular}
\normalsize

\section{Bilan du produit}

Le produit StegX répond bien aux objectifs : faire de la stéganographie 
et de la stéganalyse sur des fichiers de type image, audio et vidéo en 
utilisant plusieurs algorithmes (LSB, EOF, Metadata). 
Par ailleurs, StegX propose bien 2 interfaces : une en ligne de commande 
et une autre graphique. 
Avoir choisi le langage C a été le meilleur choix car ce langage nous a 
permis de manipuler facilement les données binaires des fichiers en entrée. 

En plus de répondre aux objectifs, l'équipe de conception s'est efforcée 
à répondre aux nombreux besoins du client, qui doit rester la personne la 
satisfaite du produit. 

\section{Conclusion sur l'organisation interne au sein du projet}

Durant le semestre 6 de la dernière année de licence informatique à Versailles, 
l'équipe StegX s'est réuni pour créer une application en rapport à la 
cryptographie. 
De ce fait, tous les membres de l'équipe de conception se sont efforcés à
travailler de façon professionnelle et ordonnée. En effet, ils leur tenaient 
à coeur de mettre en application toute la méthodologie informatique acquise 
durant la Licence. 
Par ailleurs, en raison de la grande charge de travail, il a fallu, dès le 
début du projet, de diviser la conception selon les trois différents types 
dont StegX doit se charger (image, audio et vidéo). 
Cette division de charge de travail impliquait que chacun devait réaliser 
une partie de la conception. 

Enfin, nous n'avions pas de chef de projet. Cette décision nous a été très 
favorable du fait que chacun a eu la maturité de comprendre les enjeux de 
ce grand projet. En effet, ces enjeux étaient grands et c'est pour cela 
que le projet IN608 est la matière la plus travaillée du semestre. 

\end{document}
