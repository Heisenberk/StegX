\documentclass[11pt]{article}
%\documentclass{book}
\usepackage[utf8]{inputenc}
\usepackage[T1]{fontenc}
\usepackage[french]{babel}
\usepackage[top=1.8cm, bottom=1.8cm, left=1.8cm, right=1.8cm]{geometry}
\usepackage[linktocpage,colorlinks=false]{hyperref}
\usepackage{graphicx}
\usepackage{epsfig}
\usepackage{amssymb}
\usepackage{amsmath}
\usepackage{array}
\usepackage{subfig}
\usepackage{multicol}
\usepackage{caption}
\usepackage{listings}
\usepackage{algorithm}
\usepackage{algorithmic}
\hypersetup{
    colorlinks=true,
    breaklinks=true,
    urlcolor=red,
}
\parskip=5pt

\title{\huge{\textbf Compte Rendu}}
\author{AYOUB Pierre, BASKEVITCH Claire, BESSAC Tristan, \\
CAUMES Clément, DELAUNAY Damien, DOUDOUH Yassin}
\date{Mercredi 18 Avril 2018}

\begin{document}

\maketitle
\vspace{20em}
\begin{center}\includegraphics{pictures/Application.png}\end{center}
\newpage

\tableofcontents

\newpage

\section{Introduction}

Le cahier des charges ainsi que les spécifications ont été réalisés par 
l'équipe de StegX, il est maintenant temps d'implémenter le logiciel.
L'application StegX proposera à ses utilisateurs de manipuler une interface 
graphique ou en ligne de commandes afin de cacher des données dans d'autres 
données (pour des formats pris en charge par StegX). De plus, StegX permettra 
d'extraire des données d'un fichier que l'on considère déjà comme cachant 
des données. Le logiciel proposera plusieurs algorithmes de stéganographie
tels que EOF, LSB et Metadata. 

En plus de l'implémentation, il sera utile de présenter le fonctionnement 
de l'architecture de l'application avec les explications techniques. 
Une description des points délicats seront mis en valeur ainsi qu'un 
bilan technique sur l'application et humain sur l'équipe de conception. 

\section{Architecture du produit}
\subsection{Explications techniques}

\section{Description du fonctionnement}

\section{Description des points délicats de la programmation}

\section{Comparaison estimation-implémentation}

\section{Bilan du produit}

\section{Conclusion sur l'organisation interne au sein du projet}

\end{document}
