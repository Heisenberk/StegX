\documentclass[11pt]{article}
%\documentclass{book}
\usepackage[utf8]{inputenc}
\usepackage[T1]{fontenc}
\usepackage[french]{babel}
\usepackage[top=1.8cm, bottom=1.8cm, left=1.8cm, right=1.8cm]{geometry}
\usepackage[linktocpage,colorlinks=false]{hyperref}
\usepackage{graphicx}
\usepackage{epsfig}
\usepackage{amssymb}
\usepackage{amsmath}
\usepackage{array}
\usepackage{subfig}
\usepackage{multicol}
\usepackage{caption}
\usepackage{listings}
\usepackage{algorithm}
\usepackage{algorithmic}
\usepackage{array,multirow,makecell}
\hypersetup{
    colorlinks=true,
    breaklinks=true,
    urlcolor=red,
}
\parskip=5pt

\title{\huge{\textbf Ajouts du compte rendu version du 1{er} juin}}
\author{AYOUB Pierre, BASKEVITCH Claire, BESSAC Tristan, \\
CAUMES Clément, DELAUNAY Damien, DOUDOUH Yassin}
\date{Mercredi 1 Juin 2018}

\begin{document}

\maketitle
\vspace{20em}
\begin{center}\includegraphics{pictures/Application.png}\end{center}
\newpage

\section{Introduction}

Le client nous a autorisé à rendre une nouvelle version de StegX le 1er
juin puisque celle du 25 mai ne proposait pas d'algorithmes pour le format 
MP3. En effet, le format MP3 est un format très compliqué étant compressé. 
Pour la version du 25 mai, Pierre Ayoub et Damien Delaunay avait réussi à 
étudier les versions du format (MPEG 1 Layer III, MPEG 2 Layer III) et de 
formats de métadonnée (ID3 version 1 et ID3 version 2). 

Ces études approfondies ont mené à modifier certains choix dans les spécifications 
en raison de la difficulté de ce format. 

\section{Changements des spécifications}

\subsection{Structure MP3}

\subsection{Ajouts de fonctions pour éviter la répétition de code}

\section{Bilan technique du produit pour la version du 1er juin}


\end{document}
