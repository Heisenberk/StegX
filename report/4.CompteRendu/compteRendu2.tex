\documentclass[11pt]{article}
%\documentclass{book}
\usepackage[utf8]{inputenc}
\usepackage[T1]{fontenc}
\usepackage[french]{babel}
\usepackage[top=1.8cm, bottom=1.8cm, left=1.8cm, right=1.8cm]{geometry}
\usepackage[linktocpage,colorlinks=false]{hyperref}
\usepackage{graphicx}
\usepackage{epsfig}
\usepackage{amssymb}
\usepackage{amsmath}
\usepackage{array}
\usepackage{subfig}
\usepackage{multicol}
\usepackage{caption}
\usepackage{listings}
\usepackage{algorithm}
\usepackage{algorithmic}
\usepackage{array,multirow,makecell}
\hypersetup{
    colorlinks=true,
    breaklinks=true,
    urlcolor=red,
}
\parskip=5pt

\title{\huge{\textbf Ajouts du compte rendu version du 1{er} juin}}
\author{AYOUB Pierre, BASKEVITCH Claire, BESSAC Tristan, \\
CAUMES Clément, DELAUNAY Damien, DOUDOUH Yassin}
\date{Mercredi 1 Juin 2018}

\begin{document}

\maketitle
\vspace{20em}
\begin{center}\includegraphics{pictures/Application.png}\end{center}
\newpage

\section{Introduction}

Le client nous a autorisé à rendre une nouvelle version de StegX le 1er
juin puisque celle du 25 mai ne proposait pas d'algorithme pour le format 
MP3. En effet, le MP3 est un format très compliqué étant compressé. 
Pour la version du 25 mai, Pierre Ayoub et Damien Delaunay avait réussi à 
étudier les versions du format (MPEG 1 Layer III, MPEG 2 Layer III) et de 
format de métadonnée (ID3 version 1 et ID3 version 2). 

Ces études approfondies ont mené à modifier certains choix dans les spécifications 
en raison de la difficulté de ce format. 

\section{Changements des spécifications}

\subsection{Structure MP3}

\begin{lstlisting}[language=c]
struct mp3 {
    long int fr_frst_adr; 
    long int fr_nb;       
    long int eof;      
};
\end{lstlisting}

Cette structure comporte 3 champs : 
\begin{itemize}
\item \textit{fr\_frst\_adr} est l'adresse du header de la première frame 
MPEG 1/2 Layer III. 
\item \textit{fr\_nb} est le nombre de frame MPEG 1/2 Layer III.
\item \textit{eof} est l'adresse de la fin du fichier officiel (sans signature et données cachées).
\end{itemize}

Cette structure n'a pas pu être déduite lors des spécifications puisque 
l'étude détaillée du format était trop bas niveau à cette période du projet. 
En effet, la stéganographie sur le format MP3 est très expérimentale puisque 
ce format est très complexe. 

\subsection{Ajouts de fonctions pour éviter la répétition de code}

Pour réaliser correctement le remplissage de la structure afin d'éviter la
repétition de code lors des algorithmes proposés par le format MP3, 
nous avons créé plusieurs fonctions internes au format : 

\begin{lstlisting}[language=c]
int mp3_mpeg_hdr_test(uint32_t hdr);
\end{lstlisting}

Cette fonction teste si le header hdr est un header MPEG 1/2 Layer III.
Elle renvoie 0 si le header est incorrect et 1 si le header est valide. 
\newline
\begin{lstlisting}[language=c]
int mp3_mpeg_fr_seek(uint32_t hdr, FILE * f);
\end{lstlisting}

Cette fonction va sauter la frame MP3 actuelle contenue dans le fichier f
et dont hdr est le header de la frame MP3 à sauter. Elle renvoie la valeur 
de retour de fseek. 
\newline

\begin{lstlisting}[language=c]
long int mp3_mpeg_fr_find_first(FILE * f);
\end{lstlisting}

Cette fonction trouve la première frame MPEG 1/2 Layer III dans le fichier f
MP3. Elle retourne l'adresse du header de la première frame si elle est 
trouvée, sinon -1. 
\newline

\begin{lstlisting}[language=c]
int mp3_id3v1_hdr_test(uint32_t hdr);
\end{lstlisting}

Cette fonction test si le header hdr est un header ID3v1. Elle renvoie 0 
si le header est incorrect sinon 1 si le header est valide. 

\begin{lstlisting}[language=c]
int mp3_id3v1_tag_seek(FILE * f);
\end{lstlisting}

Cette fonction saute le tag ID3v1 actuel dans le fichier MP3 f. Elle renvoie 
la valeur de retour de fseek. 

\subsection{Ajouts dûs à la résolution d'un bug}

Lors de la préparation de la démonstration, nous voulions montrer l'aspect 
multiplateforme de l'application. 
Ainsi, nous avons remarqué que, lorsque l'on fait une insertion sur un 
système d'exploitation, l'extraction doit se faire sur le même système
d'exploitation (pas forcément sur la même machine). 
Ce bug est dû au fait que la suite pseudo aléatoire générée sur Linux n'est 
pas la même que sur Windows. 
De ce fait, il a fallu créer nos propres fonctions de randomisation pour que 
la suite pseudo-aléatoire soit la même pour n'importe quelle système d'exploitation : 

\begin{lstlisting}[language=c]
void stegx_srand(unsigned int seed);
\end{lstlisting}

Cette fonction va initialiser la graine de la suite pseudo aléatoire du 
système. \textit{seed} correspond au nombre où la graine sera initialisée.
\newline 

\begin{lstlisting}[language=c]
int stegx_rand();
\end{lstlisting}

Cette fonction va renvoyer un entier pseudo-aléatoirement. 

\section{Bilan technique du produit pour la version du 1er juin}

En ce qui concerne les algorithmes proposés par StegX en fonction des 
formats pris en charge, voici le bilan de ce que propose l'application 
après la version du 1er juin: 
\newline

\begin{tabular}{|l|c|c|c|c|c|}
  \hline
  \multirow{2}*{\textbf{Format pris en charge par l'application}} & \multicolumn{5}{c|}{\textbf{Algorithmes proposés}} \\
   \cline{2-6}
    & \textbf{LSB} & \textbf{EOF} & \textbf{Metadata} 
    &\textbf{EOC} & \textbf{Junk Chunk} \\
  \hline
  \textbf{BMP} (Clément Caumes \& Yassin Doudouh) & \textbf{\checkmark} & \textbf{\checkmark} & \textbf{\checkmark} &  & \\
  \hline      
  \textbf{PNG} (Clément Caumes \& Yassin Doudouh) &  & \textbf{\checkmark} & \textbf{\checkmark} & & \\
  \hline
  \textbf{WAV} (Pierre Ayoub \& Damien Delaunay) & \textbf{\checkmark} & \textbf{\checkmark} & & & \\
  \hline 
  \textbf{MP3} (Pierre Ayoub \& Damien Delaunay) & \color{red}{\textbf{\checkmark}} & \color{red}{\textbf{\checkmark}} & & & \\
  \hline 
  \textbf{AVI} (Claire Baskevitch \& Tristan Bessac) & & & & & \textbf{\checkmark}\\
  \hline
  \textbf{FLV} (Claire Baskevitch \& Tristan Bessac) & & \textbf{\checkmark} & & \textbf{\checkmark} & \\
  \hline
\end{tabular}
\vspace{0.5cm}

On peut remarquer que la ligne pour le format MP3 est remplie. StegX propose 
maintenant de la stéganographie sur les formats MP3. 

\end{document}
