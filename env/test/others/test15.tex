\documentclass[11pt]{article}

\usepackage[utf8]{inputenc} 
\usepackage[T1]{fontenc}
\usepackage[french]{babel}
\usepackage[top=1.8cm, bottom=1.8cm, left=1.8cm, right=1.8cm]{geometry}
\usepackage{hyperref}
\hypersetup{
    colorlinks=true,
    breaklinks=true,
    urlcolor=red,
}
\parskip=5pt

\title{Stéganogaphie \& Stéganalyse}

\begin{document}

\maketitle

\section{Résumé du sujet}

La stéganographie est l'art de la dissimulation, appliquée en informatique en
cachant des données dans d'autres données. Cette dissimulation se fait
généralement au sein de fichiers multimédias. Par exemple, le fait de cacher un
mot de passe dans une image relève de la stéganographie. Plusieurs algorithmes
peuvent être utilisés, dépendant à la fois du format, de la compression et de
beaucoup d'autres paramètres. Les élèves proposeront une application permettant
de dissimuler et d'extraire des messages dans différents formats.

La stéganalyse, quant à elle, est la recherche de données cachées dans des
fichiers suspects. Si ces données sont identifiées, il faut ensuite réussir à
les extraire pour les lire.

\section{Travail à fournir}

\begin{itemize}
    \item L'application doit permettre de cacher des données dans des fichiers
        de type image, audio et vidéo.
    \item Pour la dissimulation des messages, l'application détectera le format
        du fichier contenant et proposera à l'utilisateur de choisir parmi les
        algorithmes implémentés. 
    \item Pour la réception de fichiers, on considèrera que les fichiers en
        entrée contiennent, par défaut, des données cachées. Ainsi, le
        programme fera l'extraction automatique de ces données cachées.
    \item Plusieurs formats de fichier devront être gérés par l'application,
        ainsi qu'une diversité dans les algorithmes proposés.
    \item L'application se présentera sous forme d'une bibliothèque partagée par
        deux interfaces différentes, une graphique et une en ligne de commande.
\end{itemize}

\section{Calendrier}

\begin{tabular}{|l|r|}
  \hline
  Remise du cahier des charges & Mercredi 14 mars 2018 \\
  \hline
  Présentation orale & Mercredi 21 mars 2018 \\
  \hline
  Remise des spécifications & Mercredi 18 avril 2018 \\
  \hline
  Remise du compte rendu & Vendredi 25 mai 2018 \\
  \hline
  Soutenance et démonstration & Vendredi 1 juin 2018 \\
  \hline
\end{tabular}

\vspace{1\baselineskip}

Le cahier des charges ainsi que le compte rendu devront obligatoirement être
tapés en \LaTeX.

\end{document}
